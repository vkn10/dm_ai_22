\newcommand{\student}{Заречный А.О.}


% настройки предмета
\newcommand{\lessonName}{Интеллектуальный анализ данных}
\newcommand{\workName}{Автоэнкодеры}
\newcommand{\num}{2}
\newcommand{\teacher}{Крощенко А.А.}

% настройки работы
\newcommand{\purpose}{научиться применять автоэнкодеры для осуществления визуализации данных и их анализа.}
\newcommand{\task}{ 
    \begin{itemize}
        \item Используя выборку по варианту, осуществить проецирование данных на плоскость первых двух и трех главных компонент с использованием нейросетевой модели автоэнкодера (с двумя и тремя нейронами в среднем слое);
        \item Выполнить визуализацию полученных главных компонент с использованием средств библиотеки matplotlib, обозначая экземпляры разных классов с использованием разных цветовых маркеров;
        \item Реализовать метод t-SNE для визуализации данных (использовать также 2 и 3 компонента), построить соответствующую визуализацию;
        \item Сравнить полученные результаты с анализом с использованием PCA, сделанным в ЛР №1, сделать выводы.
    \end{itemize}




    \begin{center}
        \begin{tabular}{|c|c|c|}
            \hline
            Вариант & Выборка & Класс \\
            \hline
            5&wholesale+customers.zip&Region \\
            \hline
            
        \end{tabular}
    \end{center}
}

\newcommand{\result}{
    Реализовали требуемые методы. Получили следующие результаты:


    \begin{figure}[h]
        \centering
        \includegraphics[width=0.5\textwidth]{assests/AE_2comp.png}
        \caption{\large Автоэнкодер, проекция на 2 компоненты}
    \end{figure}

    \begin{figure}[h]
        \centering
        \includegraphics[width=0.5\textwidth]{assests/AE_3comp.png}
        \caption{\large Автоэнкодер, проекция на 3 компоненты}
    \end{figure}

    \begin{figure}[h]
        \centering
        \includegraphics[width=0.5\textwidth]{assests/TSNE_2comp.png}
        \caption{\large tSNE, проекция на 2 компоненты}
    \end{figure}

    \begin{figure}[h]
        \centering
        \includegraphics[width=0.5\textwidth]{assests/TSNE_3comp.png}
        \caption{\large tSNE, проекция на 2 компоненты}
    \end{figure}
    Получили, что датасет разделим плохо.
}
\newcommand{\conclusion}{научились применять автоэнкодеры для осуществления визуализации данных и их анализа.}
